\chapter{Mantuh}
\label{ch:mantuh}
\index{meal}
\index{dumplings}
\index{meat}
\textit{Meat-filled dumplings}

Family member: Metzma Lucie

\marginnote[20pt]{\\
    \textbf{Makes 2 full pans} \\
    Prep time: 2-3 hours \\
    Cook time: 30 minutes \\
    \vspace*{\baselineskip}
    
    \textbf{Ingredients for dough} \\
    4 cups all-purpose flour \\
    1/4 cup milk \\
    1/2 tsp salt \\
    1 cup lukewarm water + 1/4 - 1/2 cup more (as much as dough takes) \\
    Oil (to work dough with hands) \\
    \vspace*{\baselineskip}
    
    \textbf{Ingredients for meat} \\
    1 medium onion, thinly chopped \\
    1 lb ground lean beef \\
    Salt \& pepper to taste \\
    Armenian or Lebanese pepper \\
}

\begin{enumerate}
    \item Chop onion and cook in a medium pot for 1-2 minutes until soft.
    \item Add lean ground beef, salt, pepper and Armenian pepper.
    \item Cook until little or no water is left from the meat.
    
    % TODO: did we want to split this up into sections?
    \item In a large bowl, mix flour and salt.
    \item Mix in milk using a spoon. Add 1 cup water and mix with hands.
    \item Add a little oil to hands to help dough unstick.
    \item Add 1/4 - 1/2 cup water slowly while mixing, until you get a soft dough.
    \item You can add up to 2 handfuls of oil.
    
    \item Flatten dough using a rolling pin or a pasta machine.
    \item Cut into 1.5-inch squares, fill with some meat and pinch sides.
    \item Place month on a baking sheet lined with parchment paper, leaving some space between each.
    \item Cook at 350\degree F for about 30 minutes, until evenly browned.
    \item Eat with yogurt, sriracha and sprinkle with sumac!
\end{enumerate}

\marginnote{
    After first cooking in oven, Sybil would normally cook them in a broth and eat as a soup. You can top with a little yogurt in there too!
    Can store cooked in bags in the freezer, reheat in oven before eating.
}
