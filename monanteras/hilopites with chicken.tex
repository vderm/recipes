\chapter{Hilopites with Chicken}
\label{ch:hilopites}
\index{meal}
\index{chicken}
\index{pasta}
\textit{Homemade egg pasta in broth with chicken}

Family member: Mom

\marginnote[20pt]{\\
    \textbf{Makes 6-8 servings} \\
    Prep time: 15-30 minutes \\
    Cook time: 2 hours \\
    \vspace*{\baselineskip}

    1 large onion \\
    1 garlic clove \\
    About 11 chicken drumsticks (without skin) \\
    Salt \\
    Pepper \\
    1/2 can tomato sauce, Hunt's (213ml) or 1 full can \\
    1 tbsp tomato paste (optional) \\
    1 cup hilopites, or orzo/pasta \\
}


\newthought{Grandma} \textgreek{Ελισσάβετ} makes \textgreek{χυλοπήτες} every summer in Greece and brings them to us when she comes back to Canada. Mom would make this dish with grandma's homemade, but you can find \textgreek{χυλοπήτες} at Greek supermarkets like Marché PA and Atlantis.

\begin{enumerate}
    \item In a large pot, grate the onion and add some olive oil. Heat on high.
    \item When it starts to sizzle, stir with a wooden spoon and add the chicken drumsticks. Season with salt and pepper and stir until chicken is lightly browned.
    \item Add the 1/2 or 1 can of tomato sauce and tomato paste (if using).
    \item Add about 3 cups of water, mix and reduce heat.
    \item Cook for 1 1/2 hours until reduced and left with about 2 cups liquid.
    \item Remove the chicken and while keeping liquid on a simmer, add the hilopites.
    \item Add another 1/2 cup water and cook on a low simmer for 10-15 minutes until the hilopites are cooked. Can add a little more water if too thick.
    \item Once hilopites are cooked and soup has thickened, remove from the heat and add back the chicken.
\end{enumerate}

\marginnote{Serve with lots of grated Romano or Parmesan. Can also make the hilopites without chicken.}